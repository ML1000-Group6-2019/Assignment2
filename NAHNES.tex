\documentclass[]{article}
\usepackage{lmodern}
\usepackage{amssymb,amsmath}
\usepackage{ifxetex,ifluatex}
\usepackage{fixltx2e} % provides \textsubscript
\ifnum 0\ifxetex 1\fi\ifluatex 1\fi=0 % if pdftex
  \usepackage[T1]{fontenc}
  \usepackage[utf8]{inputenc}
\else % if luatex or xelatex
  \ifxetex
    \usepackage{mathspec}
  \else
    \usepackage{fontspec}
  \fi
  \defaultfontfeatures{Ligatures=TeX,Scale=MatchLowercase}
\fi
% use upquote if available, for straight quotes in verbatim environments
\IfFileExists{upquote.sty}{\usepackage{upquote}}{}
% use microtype if available
\IfFileExists{microtype.sty}{%
\usepackage{microtype}
\UseMicrotypeSet[protrusion]{basicmath} % disable protrusion for tt fonts
}{}
\usepackage[margin=1in]{geometry}
\usepackage{hyperref}
\hypersetup{unicode=true,
            pdftitle={ML Assignment 2: NHANES data},
            pdfborder={0 0 0},
            breaklinks=true}
\urlstyle{same}  % don't use monospace font for urls
\usepackage{color}
\usepackage{fancyvrb}
\newcommand{\VerbBar}{|}
\newcommand{\VERB}{\Verb[commandchars=\\\{\}]}
\DefineVerbatimEnvironment{Highlighting}{Verbatim}{commandchars=\\\{\}}
% Add ',fontsize=\small' for more characters per line
\usepackage{framed}
\definecolor{shadecolor}{RGB}{248,248,248}
\newenvironment{Shaded}{\begin{snugshade}}{\end{snugshade}}
\newcommand{\AlertTok}[1]{\textcolor[rgb]{0.94,0.16,0.16}{#1}}
\newcommand{\AnnotationTok}[1]{\textcolor[rgb]{0.56,0.35,0.01}{\textbf{\textit{#1}}}}
\newcommand{\AttributeTok}[1]{\textcolor[rgb]{0.77,0.63,0.00}{#1}}
\newcommand{\BaseNTok}[1]{\textcolor[rgb]{0.00,0.00,0.81}{#1}}
\newcommand{\BuiltInTok}[1]{#1}
\newcommand{\CharTok}[1]{\textcolor[rgb]{0.31,0.60,0.02}{#1}}
\newcommand{\CommentTok}[1]{\textcolor[rgb]{0.56,0.35,0.01}{\textit{#1}}}
\newcommand{\CommentVarTok}[1]{\textcolor[rgb]{0.56,0.35,0.01}{\textbf{\textit{#1}}}}
\newcommand{\ConstantTok}[1]{\textcolor[rgb]{0.00,0.00,0.00}{#1}}
\newcommand{\ControlFlowTok}[1]{\textcolor[rgb]{0.13,0.29,0.53}{\textbf{#1}}}
\newcommand{\DataTypeTok}[1]{\textcolor[rgb]{0.13,0.29,0.53}{#1}}
\newcommand{\DecValTok}[1]{\textcolor[rgb]{0.00,0.00,0.81}{#1}}
\newcommand{\DocumentationTok}[1]{\textcolor[rgb]{0.56,0.35,0.01}{\textbf{\textit{#1}}}}
\newcommand{\ErrorTok}[1]{\textcolor[rgb]{0.64,0.00,0.00}{\textbf{#1}}}
\newcommand{\ExtensionTok}[1]{#1}
\newcommand{\FloatTok}[1]{\textcolor[rgb]{0.00,0.00,0.81}{#1}}
\newcommand{\FunctionTok}[1]{\textcolor[rgb]{0.00,0.00,0.00}{#1}}
\newcommand{\ImportTok}[1]{#1}
\newcommand{\InformationTok}[1]{\textcolor[rgb]{0.56,0.35,0.01}{\textbf{\textit{#1}}}}
\newcommand{\KeywordTok}[1]{\textcolor[rgb]{0.13,0.29,0.53}{\textbf{#1}}}
\newcommand{\NormalTok}[1]{#1}
\newcommand{\OperatorTok}[1]{\textcolor[rgb]{0.81,0.36,0.00}{\textbf{#1}}}
\newcommand{\OtherTok}[1]{\textcolor[rgb]{0.56,0.35,0.01}{#1}}
\newcommand{\PreprocessorTok}[1]{\textcolor[rgb]{0.56,0.35,0.01}{\textit{#1}}}
\newcommand{\RegionMarkerTok}[1]{#1}
\newcommand{\SpecialCharTok}[1]{\textcolor[rgb]{0.00,0.00,0.00}{#1}}
\newcommand{\SpecialStringTok}[1]{\textcolor[rgb]{0.31,0.60,0.02}{#1}}
\newcommand{\StringTok}[1]{\textcolor[rgb]{0.31,0.60,0.02}{#1}}
\newcommand{\VariableTok}[1]{\textcolor[rgb]{0.00,0.00,0.00}{#1}}
\newcommand{\VerbatimStringTok}[1]{\textcolor[rgb]{0.31,0.60,0.02}{#1}}
\newcommand{\WarningTok}[1]{\textcolor[rgb]{0.56,0.35,0.01}{\textbf{\textit{#1}}}}
\usepackage{graphicx,grffile}
\makeatletter
\def\maxwidth{\ifdim\Gin@nat@width>\linewidth\linewidth\else\Gin@nat@width\fi}
\def\maxheight{\ifdim\Gin@nat@height>\textheight\textheight\else\Gin@nat@height\fi}
\makeatother
% Scale images if necessary, so that they will not overflow the page
% margins by default, and it is still possible to overwrite the defaults
% using explicit options in \includegraphics[width, height, ...]{}
\setkeys{Gin}{width=\maxwidth,height=\maxheight,keepaspectratio}
\IfFileExists{parskip.sty}{%
\usepackage{parskip}
}{% else
\setlength{\parindent}{0pt}
\setlength{\parskip}{6pt plus 2pt minus 1pt}
}
\setlength{\emergencystretch}{3em}  % prevent overfull lines
\providecommand{\tightlist}{%
  \setlength{\itemsep}{0pt}\setlength{\parskip}{0pt}}
\setcounter{secnumdepth}{0}
% Redefines (sub)paragraphs to behave more like sections
\ifx\paragraph\undefined\else
\let\oldparagraph\paragraph
\renewcommand{\paragraph}[1]{\oldparagraph{#1}\mbox{}}
\fi
\ifx\subparagraph\undefined\else
\let\oldsubparagraph\subparagraph
\renewcommand{\subparagraph}[1]{\oldsubparagraph{#1}\mbox{}}
\fi

%%% Use protect on footnotes to avoid problems with footnotes in titles
\let\rmarkdownfootnote\footnote%
\def\footnote{\protect\rmarkdownfootnote}

%%% Change title format to be more compact
\usepackage{titling}

% Create subtitle command for use in maketitle
\providecommand{\subtitle}[1]{
  \posttitle{
    \begin{center}\large#1\end{center}
    }
}

\setlength{\droptitle}{-2em}

  \title{ML Assignment 2: NHANES data}
    \pretitle{\vspace{\droptitle}\centering\huge}
  \posttitle{\par}
    \author{true \\ true \\ true \\ true \\ true}
    \preauthor{\centering\large\emph}
  \postauthor{\par}
      \predate{\centering\large\emph}
  \postdate{\par}
    \date{Oct 6, 2019}

% Any extra LaTeX you need in the preamble

\begin{document}
\maketitle

\newpage

\hypertarget{introduction}{%
\section{Introduction:}\label{introduction}}

The following is a hypothetical business problem: A pharmaceutical
company is looking to better understand what the data related to
subjects and various health conditions and miscellaneous attributes.

\begin{itemize}
\item
  \textbf{Terms}

  \textbf{Subject} - is a person who has been surveyed by the NHMS
  dataset for various attributes related to the following: demographics,
  examinations, dietary, questionnaire(medical conditions), and
  medication

  \textbf{Health Conditions} - various diseases or ailments that people
  may inhibit such as sleep disorders, diabetes, oral health,
  cholesterol.

  \textbf{The National Health and Nutrition Examination Survey (NHANES)}
  - is a program of studies designed to assess the health and
  nutritional status of adults and children in the
\item
  \textbf{Data Description}
\end{itemize}

The data is spread against 6 spreadsheets (CSV): Demographics,
Examinations, Dietary, Laboratory, Questionnaire, and Medication.

\hypertarget{business-case}{%
\section{Business Case}\label{business-case}}

A pharmaceutical company wants to produce new drugs. The company is
curious as to whether existing data on subjects and their associated
health conditions could provide advice and insight to their drug
researchers. They have obtained NHMS dataset. This dataset contains
subject/patient data along with various information including health
conditions.

The company is interested in producing new drugs for the following
health conditions: diabetes and hypertension/cholesterol (we can add or
remove health conditions later. At the very least, let's keep diabetes
or something).

\hypertarget{first-problem}{%
\subsection{First Problem:}\label{first-problem}}

What types of symptoms, medications, diet, demographics are common among
various health conditions (such as diabetes)? For example, what types of
dietary factors are commonly found with diabetes?

\hypertarget{second-problem}{%
\subsection{Second Problem:}\label{second-problem}}

Are their any commonalities between various people with the same health
conditions? For example, if subject 1 and subject 2 have the same health
condition (for example, diabetes) what other similarities would these
subjects have?

They have approached our Machine Learning group for help on these
problems.

\hypertarget{analytical-reframing-for-the-business-case.}{%
\section{Analytical Reframing for the Business
Case.}\label{analytical-reframing-for-the-business-case.}}

The first business problem involves using ``health condition'' features
and finding related features. This is an association problem and we will
need a model using an association algorithm.

\begin{itemize}
\item
  What features are associated with ``health condition'' features?
\item
  We're comparing the rows of the dataset.
\end{itemize}

The second business problem involves finding commonality between
subjects. This is a clustering problem and we will need a model using a
clustering algorithm. We need to determine whether business's
presumption is accurate:

\begin{itemize}
\item
  Can subjects be divided into discrete groups according to their health
  conditions, which could then provide meaningful data for the drug
  researchers? If yes, we need to find these clusters of subjects that
  can be used to segregate the data by health conditions, then we can
  report these findings to the business.
\item
  Or is there too much commonality between ``health condition'' features
  and other features?
\item
  If we cannot find clusters that could be divided by health conditions,
  then we will also report this finding to the company as well and note
  that clusters that we did find. We're comparing the columns/attributes
  of the dataset.
\end{itemize}

\hypertarget{how-do-we-define-health-conditions-within-the-dataset}{%
\subsection{How do we define health conditions within the dataset
?}\label{how-do-we-define-health-conditions-within-the-dataset}}

How do we know what is considered a health condition within the data? We
could use lab data and diagnose whether someone falls under the
definition of health condition; however, for the purpose

Diabetes within Questionnaire dataset:

We are postulating that the following features/columns indicate that an
individual has diabetes

DID040 - ``How old \{was SP/were you\} when a doctor or other health
professional first told \{you/him/her\} that \{you/he/she\} had diabetes
or sugar diabetes?''

DID060 - ``For how long \{have you/has SP\} been taking insulin?''

NA in the above features might indicate the subject does not have
diabetes.

For the association problem, we will need to see which attributes are
tied to above features.

Blood Pressure within Questionnaire dataset

BPD035 - How old \{were you/was SP\} when \{you were/he/she was\} first
told that \{you/he/she\} had hypertension or high blood pressure?

BPQ020 - \{Have you/Has SP\} ever been told by a doctor or other health
professional that \{you/s/he\} had hypertension, also called high blood
pressure?

Cancer

MCQ220 - \{Have you/Has SP\} ever been told by a doctor or other health
professional that \{you/s/he\} had cancer or a malignancy
(ma-lig-nan-see) of any kind?

\hypertarget{loading-r-packages}{%
\section{Loading R packages}\label{loading-r-packages}}

\begin{Shaded}
\begin{Highlighting}[]
\KeywordTok{library}\NormalTok{(plyr)}
\KeywordTok{library}\NormalTok{(dplyr)}
\KeywordTok{library}\NormalTok{(tidyr)}
\KeywordTok{library}\NormalTok{(ggplot2)}
\KeywordTok{library}\NormalTok{(knitr)}
\end{Highlighting}
\end{Shaded}

\hypertarget{data-cleaning}{%
\section{Data cleaning}\label{data-cleaning}}

As indicated eariler, the dataset conists of 6 raw data files:
Demographics, Examinations, Dietary, Laboratory, Questionnaire, and
Medication. The largest file (in terms of attributes) is the
questionnaire data contains 953 variables; while the smallest data file
contains 47 variables. This is a large amount of data. Cumatavitely,
there are 1000s of attributes, therefore, we have decided need to employ
the following guidelines to assist with reducing the complexity of the
data:

\begin{itemize}
\tightlist
\item
  If more than 25\% of the values are missing for an attribute(column),
  we will consider removing the column from further evaluation.
\item
  If the majority of attributes are missing 25\% or more of their values
  for a given dataset, we will use personal or business judgmement to
  subjectively select a smaller subset of ``interesting'' values. Of
  these subset of missing values, we will then decide how to impute the
  values on these attributes.
\end{itemize}

Ideally, we would like to analyse and impute every attribute with
missing values, but in this situation, it may not be practical due to
the large volume of missing data.

\begin{Shaded}
\begin{Highlighting}[]
\CommentTok{# Reading files}
\NormalTok{demographic   =}\StringTok{ }\KeywordTok{read.csv}\NormalTok{(}\StringTok{"Data/Raw/demographic.csv"}\NormalTok{, }\DataTypeTok{header =} \OtherTok{TRUE}\NormalTok{, }\DataTypeTok{na.strings =} \KeywordTok{c}\NormalTok{(}\StringTok{"NA"}\NormalTok{,}\StringTok{""}\NormalTok{,}\StringTok{"#NA"}\NormalTok{))}
\NormalTok{diet          =}\StringTok{ }\KeywordTok{read.csv}\NormalTok{(}\StringTok{"Data/Raw/diet.csv"}\NormalTok{, }\DataTypeTok{header =} \OtherTok{TRUE}\NormalTok{, }\DataTypeTok{na.strings =} \KeywordTok{c}\NormalTok{(}\StringTok{"NA"}\NormalTok{,}\StringTok{""}\NormalTok{,}\StringTok{"#NA"}\NormalTok{))}
\NormalTok{examination   =}\StringTok{ }\KeywordTok{read.csv}\NormalTok{(}\StringTok{"Data/Raw/examination.csv"}\NormalTok{, }\DataTypeTok{header =} \OtherTok{TRUE}\NormalTok{, }\DataTypeTok{na.strings =} \KeywordTok{c}\NormalTok{(}\StringTok{"NA"}\NormalTok{,}\StringTok{""}\NormalTok{,}\StringTok{"#NA"}\NormalTok{))}
\NormalTok{labs          =}\StringTok{ }\KeywordTok{read.csv}\NormalTok{(}\StringTok{"Data/Raw/labs.csv"}\NormalTok{, }\DataTypeTok{header =} \OtherTok{TRUE}\NormalTok{, }\DataTypeTok{na.strings =} \KeywordTok{c}\NormalTok{(}\StringTok{"NA"}\NormalTok{,}\StringTok{""}\NormalTok{,}\StringTok{"#NA"}\NormalTok{))}
\NormalTok{medications   =}\StringTok{ }\KeywordTok{read.csv}\NormalTok{(}\StringTok{"Data/Raw/medications.csv"}\NormalTok{, }\DataTypeTok{header =} \OtherTok{TRUE}\NormalTok{, }\DataTypeTok{na.strings =} \KeywordTok{c}\NormalTok{(}\StringTok{"NA"}\NormalTok{,}\StringTok{""}\NormalTok{,}\StringTok{"#NA"}\NormalTok{))}
\NormalTok{questionnaire =}\StringTok{ }\KeywordTok{read.csv}\NormalTok{(}\StringTok{"Data/Raw/questionnaire.csv"}\NormalTok{, }\DataTypeTok{header =} \OtherTok{TRUE}\NormalTok{, }\DataTypeTok{na.strings =} \KeywordTok{c}\NormalTok{(}\StringTok{"NA"}\NormalTok{,}\StringTok{""}\NormalTok{,}\StringTok{"#NA"}\NormalTok{))}

\CommentTok{# Merging files}
\NormalTok{data_List =}\StringTok{ }\KeywordTok{list}\NormalTok{(demographic,examination,diet,labs,questionnaire,medications)}
\NormalTok{Data_joined =}\StringTok{ }\KeywordTok{join_all}\NormalTok{(data_List) }\CommentTok{#require(plyr)}
\end{Highlighting}
\end{Shaded}

\hypertarget{checking-for-missing-data}{%
\subsection{Checking for missing data}\label{checking-for-missing-data}}

Its always important to check for missing values and consider how to fix
them.

\begin{itemize}
\tightlist
\item
  \textbf{Demographic}
\end{itemize}

\begin{Shaded}
\begin{Highlighting}[]
\KeywordTok{library}\NormalTok{(plotly)}
\end{Highlighting}
\end{Shaded}

\begin{Shaded}
\begin{Highlighting}[]
\NormalTok{demographic_MS <-}\StringTok{ }\NormalTok{demographic }\OperatorTok\StringTok{ }\KeywordTok{summarise_all}\NormalTok{(}\OperatorTok{~}\NormalTok{(}\KeywordTok{sum}\NormalTok{(}\KeywordTok{is.na}\NormalTok{(.))}\OperatorTok{/}\KeywordTok{n}\NormalTok{()))}
\NormalTok{demographic_MS <-}\StringTok{ }\KeywordTok{gather}\NormalTok{(demographic_MS, }\DataTypeTok{key =} \StringTok{"variables"}\NormalTok{, }\DataTypeTok{value =} \StringTok{"percent_missing"}\NormalTok{)}
\NormalTok{demographic_MS <-}\StringTok{ }\NormalTok{demographic_MS[demographic_MS}\OperatorTok{$}\NormalTok{percent_missing }\OperatorTok{>}\StringTok{ }\FloatTok{0.0}\NormalTok{, ] }

\NormalTok{demographic_MS_plot  <-}\StringTok{ }\KeywordTok{ggplot}\NormalTok{(demographic_MS, }\KeywordTok{aes}\NormalTok{(}\DataTypeTok{x =} \KeywordTok{reorder}\NormalTok{(variables, percent_missing), }\DataTypeTok{y =}\NormalTok{ percent_missing)) }\OperatorTok{+}
\StringTok{  }\KeywordTok{geom_bar}\NormalTok{(}\DataTypeTok{stat =} \StringTok{"identity"}\NormalTok{, }\DataTypeTok{fill =} \StringTok{"blue"}\NormalTok{, }\KeywordTok{aes}\NormalTok{(}\DataTypeTok{color =} \KeywordTok{I}\NormalTok{(}\StringTok{'white'}\NormalTok{)), }\DataTypeTok{size =} \FloatTok{0.3}\NormalTok{, }\DataTypeTok{alpha =} \FloatTok{0.8}\NormalTok{)}\OperatorTok{+}
\StringTok{  }\KeywordTok{xlab}\NormalTok{(}\StringTok{'variables'}\NormalTok{)}\OperatorTok{+}
\StringTok{  }\KeywordTok{coord_flip}\NormalTok{()}\OperatorTok{+}\StringTok{ }
\StringTok{  }\CommentTok{#theme_fivethirtyeight() +}
\StringTok{  }\KeywordTok{ggtitle}\NormalTok{(}\StringTok{"Demographic Missing Data By Columns"}\NormalTok{)}

\NormalTok{demographic_MS_plot}
\end{Highlighting}
\end{Shaded}

\includegraphics{NAHNES_files/figure-latex/unnamed-chunk-5-1.pdf}

\begin{itemize}
\tightlist
\item
  \textbf{Medications}
\end{itemize}

\begin{Shaded}
\begin{Highlighting}[]
\NormalTok{medications_MS <-}\StringTok{ }\NormalTok{medications }\OperatorTok\StringTok{ }\KeywordTok{summarise_all}\NormalTok{(}\OperatorTok{~}\NormalTok{(}\KeywordTok{sum}\NormalTok{(}\KeywordTok{is.na}\NormalTok{(.))}\OperatorTok{/}\KeywordTok{n}\NormalTok{()))}
\NormalTok{medications_MS <-}\StringTok{ }\KeywordTok{gather}\NormalTok{(medications_MS, }\DataTypeTok{key =} \StringTok{"variables"}\NormalTok{, }\DataTypeTok{value =} \StringTok{"percent_missing"}\NormalTok{)}
\NormalTok{medications_MS <-}\StringTok{ }\NormalTok{medications_MS[medications_MS}\OperatorTok{$}\NormalTok{percent_missing }\OperatorTok{>}\StringTok{ }\FloatTok{0.0}\NormalTok{, ] }

\NormalTok{medications_MS_plot <-}\StringTok{ }\KeywordTok{ggplot}\NormalTok{(medications_MS, }\KeywordTok{aes}\NormalTok{(}\DataTypeTok{x =} \KeywordTok{reorder}\NormalTok{(variables, percent_missing), }\DataTypeTok{y =}\NormalTok{ percent_missing)) }\OperatorTok{+}
\StringTok{  }\KeywordTok{geom_bar}\NormalTok{(}\DataTypeTok{stat =} \StringTok{"identity"}\NormalTok{, }\DataTypeTok{fill =} \StringTok{"blue"}\NormalTok{, }\KeywordTok{aes}\NormalTok{(}\DataTypeTok{color =} \KeywordTok{I}\NormalTok{(}\StringTok{'white'}\NormalTok{)), }\DataTypeTok{size =} \FloatTok{0.3}\NormalTok{, }\DataTypeTok{alpha =} \FloatTok{0.8}\NormalTok{)}\OperatorTok{+}
\StringTok{  }\KeywordTok{xlab}\NormalTok{(}\StringTok{'variables'}\NormalTok{)}\OperatorTok{+}
\StringTok{  }\KeywordTok{coord_flip}\NormalTok{()}\OperatorTok{+}\StringTok{ }
\StringTok{  }\CommentTok{#theme_fivethirtyeight() +}
\StringTok{  }\KeywordTok{ggtitle}\NormalTok{(}\StringTok{"Medications Missing Data By Columns"}\NormalTok{)}

\NormalTok{medications_MS_plot}
\end{Highlighting}
\end{Shaded}

\includegraphics{NAHNES_files/figure-latex/unnamed-chunk-6-1.pdf}

\begin{itemize}
\tightlist
\item
  \textbf{Others spreadsheets }
\end{itemize}

We didn't represent the rest of spreadsheets because the percentage of
missing data are very important. Just for example

\begin{Shaded}
\begin{Highlighting}[]
\NormalTok{examination_MS <-}\StringTok{ }\NormalTok{examination }\OperatorTok\StringTok{ }\KeywordTok{summarise_all}\NormalTok{(}\OperatorTok{~}\NormalTok{(}\KeywordTok{sum}\NormalTok{(}\KeywordTok{is.na}\NormalTok{(.))}\OperatorTok{/}\KeywordTok{n}\NormalTok{()))}
\NormalTok{examination_MS <-}\StringTok{ }\KeywordTok{gather}\NormalTok{(examination_MS, }\DataTypeTok{key =} \StringTok{"variables"}\NormalTok{, }\DataTypeTok{value =} \StringTok{"percent_missing"}\NormalTok{)}
\NormalTok{examination_MS <-}\StringTok{ }\NormalTok{examination_MS[examination_MS}\OperatorTok{$}\NormalTok{percent_missing }\OperatorTok{>}\StringTok{ }\FloatTok{0.0}\NormalTok{, ] }

\NormalTok{examination_MS_plot <-}\StringTok{ }\KeywordTok{ggplot}\NormalTok{(examination_MS, }\KeywordTok{aes}\NormalTok{(}\DataTypeTok{x =} \KeywordTok{reorder}\NormalTok{(variables, percent_missing), }\DataTypeTok{y =}\NormalTok{ percent_missing)) }\OperatorTok{+}
\StringTok{  }\KeywordTok{geom_bar}\NormalTok{(}\DataTypeTok{stat =} \StringTok{"identity"}\NormalTok{, }\DataTypeTok{fill =} \StringTok{"blue"}\NormalTok{, }\KeywordTok{aes}\NormalTok{(}\DataTypeTok{color =} \KeywordTok{I}\NormalTok{(}\StringTok{'white'}\NormalTok{)), }\DataTypeTok{size =} \FloatTok{0.3}\NormalTok{, }\DataTypeTok{alpha =} \FloatTok{0.8}\NormalTok{)}\OperatorTok{+}
\StringTok{  }\KeywordTok{xlab}\NormalTok{(}\StringTok{'variables'}\NormalTok{)}\OperatorTok{+}
\StringTok{  }\KeywordTok{coord_flip}\NormalTok{()}\OperatorTok{+}\StringTok{ }
\StringTok{  }\CommentTok{#theme_fivethirtyeight() +}
\StringTok{  }\KeywordTok{ggtitle}\NormalTok{(}\StringTok{"Examination Missing Data By Columns"}\NormalTok{)}

\NormalTok{examination_MS_plot}
\end{Highlighting}
\end{Shaded}

\includegraphics{NAHNES_files/figure-latex/unnamed-chunk-7-1.pdf}

** Diet

diet\_MS \textless- diet \%\textgreater\%
summarise\_all(\textasciitilde(sum(is.na(.))/n())) diet\_MS \textless-
gather(diet\_MS, key = ``variables'', value = ``percent\_missing'')
diet\_MS \textless- diet\_MS{[}diet\_MS\$percent\_missing \textgreater{}
0.0, {]} ggplot(diet\_MS, aes(x = reorder(variables, percent\_missing),
y = percent\_missing)) + geom\_bar(stat = ``identity'', fill = ``blue'',
aes(color = I(`white')), size = 0.3, alpha = 0.8)+ xlab(`variables')+
coord\_flip()+ \#theme\_fivethirtyeight() + ggtitle(``Diet Missing Data
By Columns'')

Almost all columns/attributes have varying degrees of missing values. As
per our guidelines, we will choose and select ``interesting''
attributes/columns based on our business/personal judgements. The NHANES
data dictionary/variable list is listed as follows:

\url{https://wwwn.cdc.gov/Nchs/Nhanes/Search/variablelist.aspx?Component=Dietary\&CycleBeginYear=2013}

The data is divded into the following sections:

We have inspected the above data dictionary and we have selected the
following interesting values for further analysis.

diet\_subset -\textgreater{} Need to impute it.

These variables/attributes are referring to the one day nutrituion
numerical attributes.

\hypertarget{data-splitting-imputation}{%
\subsection{Data splitting \&
imputation}\label{data-splitting-imputation}}

There are many ways to do data imputation, but random forest imputation
will be used since it is robust and reliable method.

\hypertarget{visualising-all-numeric-columns}{%
\subsection{Visualising all numeric
columns}\label{visualising-all-numeric-columns}}

It is useful to show histograms of all numeric columns.

\hypertarget{visualising-correlation}{%
\subsection{Visualising correlation}\label{visualising-correlation}}

\hypertarget{exploring-by-location}{%
\subsection{Exploring by location}\label{exploring-by-location}}

\hypertarget{exploring-by-age}{%
\subsection{Exploring by age}\label{exploring-by-age}}

\hypertarget{problem-1-clustering}{%
\section{Problem 1: Clustering}\label{problem-1-clustering}}

\hypertarget{pca}{%
\subsection{PCA}\label{pca}}

\hypertarget{k-means}{%
\subsection{K-means}\label{k-means}}

\hypertarget{hierarchical-agglomerative}{%
\subsection{Hierarchical
Agglomerative}\label{hierarchical-agglomerative}}

\hypertarget{summary-of-models}{%
\subsection{Summary of models}\label{summary-of-models}}

\hypertarget{problem-2-association}{%
\section{Problem 2: Association}\label{problem-2-association}}


\end{document}
